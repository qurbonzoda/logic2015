\documentclass[12pt,a4paper,oneside]{scrartcl}
\usepackage[utf8]{inputenc}
\usepackage[english,russian]{babel}
\usepackage[top=1cm,bottom=1cm,left=1cm,right=1cm]{geometry}

\begin{document}
\pagestyle{empty}

\begin{center}
{\large\scshape\bfseries Список вопросов к курсу <<Математическая логика>>}\\
ИТМО, группы M3236--M3239, осень 2015 г.
\end{center}

\vspace{0.5cm}

\begin{enumerate}
\item Исчисление высказываний. Общезначимость, доказуемость и выводимость. Корректность, полнота, непротиворечивость.
Теорема о дедукции для исчисления высказываний.
Теорема о полноте исчисления высказываний.
\item Интуиционистское исчисление высказываний. BHK-интерпретация. Непротиворечивость классического
исчисления высказываний относительно интуиционистского, теорема Гливенко.
\item Булевы и псевдобулевы алгебры. Алгебра Линденбаума. Полнота интуиционистского исчисления 
высказываний в псевдобулевых алгебрах.
\item Модели Крипке. Сведение моделей Крипке к псевдобулевым алгебрам. 
Гёделева алгебра. Нетабличность и дизъюнктивность интуиционистского исчисления высказываний.
\item Исчисление предикатов. Общезначимость и выводимость. Теорема о дедукции в исчислении предикатов.
\item Непротиворечивые множества формул. Доказательство существования моделей у непротиворечивых множеств формул 
в бескванторном исчислении предикатов.
\item Теорема Гёделя о полноте исчисления предикатов. Доказательство полноты исчисления предикатов.
\item Неразрешимость исчисления предикатов.
\item Теории первого порядка, структуры и модели. Аксиоматика Пеано. Арифметические операции. Формальная арифметика. 
\item Рекурсивные функции и отношения. Функция Аккермана. Существование рекурсивных функций,
не являющихся примитивно-рекурсивными. 
\item Выразимость отношений и представимость функций в формальной арифметике. Бета-функция Гёделя. 
Представимость рекурсивных функций в формальной арифметике.
\item Гёделева нумерация. Выводимость и рекурсивные функции.
\item Непротиворечивость и $\omega$-непротиворечивость. Первая теорема Гёделя о неполноте арифметики.
Формулировка первой теоремы Гёделя о неполноте арифметики в форме Россера.
\item Условия выводимости Гильберта-Бернайса. Формулировка второй теоремы Гёделя о неполноте арифметики, 
$Consis$. Существенность условий выводимости Гильберта-Бернайса. 
\item Вторая теорема Гёделя о неполноте арифметики. Теорема Тарского о невыразимости истины.
\item Теория множеств. Аксиоматика Цермело-Френкеля.
\item Вполне упорядоченные множества. Аксиома выбора. Альтернативные формулировки (лемма Цорна, принцип 
максимума Хаусдорфа, теорема Цермело, существование обратной для сюрь\-ективной функции). Критика аксиомы выбора.
\item Ординальные числа. Операции над ординальными числами. Мощность множеств, кардинальные числа. Теорема
Кантора-Бернштейна. Трансфинитная индукция.
\item Теорема Лёвенгейма-Сколема. Парадокс Сколема.
\item Исчисление $S_\infty$. Сведение непротиворечивости формальной арифметики к не\-про\-ти\-воречивости $S_\infty$.
\item Устранение сечений в $S_\infty$. Доказательство непротиворечивости формальной арифметики.
\end{enumerate}

\end{document}