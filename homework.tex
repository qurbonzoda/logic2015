\documentclass[11pt,a4paper,oneside]{article}
\usepackage[utf8]{inputenc}
\usepackage[english,russian]{babel}
\usepackage{amssymb}
%\usepackage{amsmath}
%\usepackage{mathabx}
\usepackage[left=2cm,right=2cm,top=2cm,bottom=2cm,bindingoffset=0cm]{geometry}
\usepackage{bnf}
\newcommand{\lit}[1]{\mbox{`\texttt{#1}'}}
\newcommand{\ntm}[1]{<\mbox{#1}>}
\begin{document}


%\title{Домашние задания по курсу <<Математическая логика>>}
%\author{Andrew Roberts}
%\date{ИТМО, группы 2536-2539, осень 2015 г.}
%\maketitle
\begin{center}
\begin{Large}{\bfseries Домашние задания по курсу <<Математическая логика>>}\end{Large}\\
\vspace{1mm}
\begin{small} ИТМО, группы M3236..M3239\end{small}\\
\small Осень 2015 г.
\end{center}

%\renewcommand{\abstractname}{Общие замечания}
%\begin{abstract}
\subsection*{Общие замечания}
Для всех программ кодировка входных и выходных файлов должна быть либо CP1251,
либо UTF8. 
Задания 1-3 --- для всех групп. 
%\end{abstract}

\subsection*{Задача 1. Проверка вывода}
Написать программу, проверяющую вывод $\gamma_1, \dots \gamma_n \vdash \alpha$ в исчислении высказываний на 
корректность. В заголовке должны быть перечислены предположения $\gamma_i$ (этот список может быть пустым) и 
доказываемое утверждение $\alpha$. В последующих строках должны быть указаны выражения вывода.
\begin{bnf}\begin{eqnarray*}
\ntm{файл} &::=& \ntm{заголовок} \lit{\textbackslash{}n} \{ \ntm{выражение} \lit{\textbackslash{}n} \}^*\\
\ntm{заголовок} &::=& \left[\ntm{выражение} \left\{ \lit{,}\ntm{выражение}\right\}^*\right] \lit{|-} \ntm{выражение}\\
\ntm{выражение} &::=& \ntm{дизъюнкция} | \ntm{дизъюнкция} \lit{->} \ntm{выражение}\\
\ntm{дизъюнкция} &::=& \ntm{конъюнкция} | \ntm{дизъюнкция} \lit{|} \ntm{конъюнкция}\\
\ntm{конъюнкция} &::=& \ntm{отрицание} | \ntm{конъюнкция} \lit{\&} \ntm{отрицание}\\
\ntm{отрицание} &::=& (\lit{A} \dots \lit{Z}) \{\lit{A}\dots\lit{Z}|\lit{0}\dots\lit{9}\}^* | \lit{!} \ntm{отрицание} | \lit{(} \ntm{выражение} \lit{)}
\end{eqnarray*}\end{bnf}%

Пробелы, символы табуляции и возврата каретки (CR) должны игнорироваться. 
Символ `\texttt{|}' имеет ASCII-код $124_{10}$.

Результатом работы программы должен быть проаннотированный текст доказательства,
в котором первая строка --- это заголовок из входного файла, каждая же последующая строка ---
соответствующая строка из вывода, расширенная в соответствии с грамматикой:
\begin{bnf}\begin{eqnarray*}
\ntm{строка} &::=& \lit{(} \ntm{номер} \lit{) } \ntm{выражение} \lit{ (} \ntm{аннотация} \lit{)}\\
\ntm{аннотация} &::=& \lit{Сх. акс. } \ntm{номер} \\
		&|& \lit{Предп. } \ntm{номер}\\
                &|& \lit{M.P. } \ntm{номер}\lit{, }\ntm{номер}\\
                &|& \lit{Не доказано}\\
\ntm{номер} &::=& \{\lit{0}\dots\lit{9}\}^+
\end{eqnarray*}\end{bnf}%

Выражение не должно содержать пробелов, номер от выражения и выражение от аннотации должны
отделяться одним пробелом. Выражения в доказательстве должны нумероваться подряд
натуральными числами с 1. Если выражение $\delta_n$ получено из 
$\delta_i$ и $\delta_j$, где $\delta_j \equiv \delta_i\rightarrow\delta_n$
путём применения правила Modus Ponens, то аннотация должна выглядеть как 
\lit{M.P. $i$, $j$}, обратный порядок номеров не допускается.

Уделите внимание производительности: ваша программа должна проверять доказательство в 
5000 выражений (общим объемом $1$Мб) на Intel Core i5-2520M ($2.5$ GHz) за несколько секунд.

\subsection*{Задача 2. Теорема о дедукции}
Написать программу, преобразующую вывод $\Gamma, \alpha \vdash \beta$ в вывод
$\Gamma \vdash \alpha \rightarrow \beta$.
Входной файл удовлетворяет грамматике из предыдущего задания,
в заголовке обязательно должно присутствовать как минимум одно предположение.

Результатом работы программы должен быть текст, содержащий преобразованный вывод.
Формат выходного файла совпадает с форматом входного файла.
Вы можете предполагать, что входной файл содержит корректный вывод требуемой формулы.

\subsection*{Задача 3. Теорема о полноте исчисления высказываний}
Написать программу, строящую доказательство указанного во входном файле высказывания
(если оно общезначимо), либо дающую оценку пропозициональных переменных, на которых
высказывание ложно (если оно опровержимо).

Входной файл состоит из единственной строки, содержащей формулу исчисления высказываний, которую
требуется доказать или опровергнуть. Высказывание удовлетворяет грамматике из первого задания.
Выходной файл должен либо содержать доказательство высказывания (в формате доказательства из 
первого задания), либо содержать фразу, удовлетворяющую грамматике:
\begin{bnf}\begin{eqnarray*}
\ntm{строка} &::=& \lit{Высказывание ложно при } ~\ntm{назначение} \{\lit{,} \ntm{назначение} \}^*\\
\ntm{назначение} &::=& \ntm{переменная} \lit{=} (\lit{И}|\lit{Л})
\end{eqnarray*}\end{bnf}%
Например, при входной формуле \texttt{!A\&!B} результат (с точностью до порядка переменных
и конкретного контрпримера) должен выглядеть так:
\begin{verbatim}
Высказывание ложно при A=И, B=Л
\end{verbatim}


\end{document}
